\documentclass[DM,authoryear,toc]{lsstdoc}
% lsstdoc documentation: https://lsst-texmf.lsst.io/lsstdoc.html
\input{meta}

% Package imports go here.

% Local commands go here.

%If you want glossaries
%\input{aglossary.tex}
%\makeglossaries

\title{Verification Tests of the DP0.1 TAPserver on IDF}

% Optional subtitle
% \setDocSubtitle{A subtitle}

\author{%
Douglas Tucker
}

\setDocRef{RTN-026}
\setDocUpstreamLocation{\url{https://github.com/lsst/rtn-026}}

\date{\vcsDate}

% Optional: name of the document's curator
% \setDocCurator{The Curator of this Document}

\setDocAbstract{%
DP0.1 contains the DWF data set from the Vera C. Rubin LSST DESC DR2.  This data set contains, among other material, images and catalogs for c. 300 sq deg of contiguous sky within the LSST footprint.  Here, we present verfication tests of the catalog data contained within the DP0.1 TAPserver on the Interim Data Facility (IDF).
}

% Change history defined here.
% Order: oldest first.
% Fields: VERSION, DATE, DESCRIPTION, OWNER NAME.
% See LPM-51 for version number policy.
\setDocChangeRecord{%
  \addtohist{1}{YYYY-MM-DD}{Unreleased.}{Douglas Tucker}
}


\begin{document}

% Create the title page.
\maketitle
% Frequently for a technote we do not want a title page  uncomment this to remove the title page and changelog.
% use \mkshorttitle to remove the extra pages

% ADD CONTENT HERE
% You can also use the \input command to include several content files.

\appendix
% Include all the relevant bib files.
% https://lsst-texmf.lsst.io/lsstdoc.html#bibliographies
\section{References} \label{sec:bib}
\renewcommand{\refname}{} % Suppress default Bibliography section
\bibliography{local,lsst,lsst-dm,refs_ads,refs,books}

% Make sure lsst-texmf/bin/generateAcronyms.py is in your path
\section{Acronyms} \label{sec:acronyms}
\addtocounter{table}{-1}
\begin{longtable}{p{0.145\textwidth}p{0.8\textwidth}}\hline
\textbf{Acronym} & \textbf{Description}  \\\hline

CPU & Central Processing Unit \\\hline
DESC & Dark Energy Science Collaboration \\\hline
DM & Data Management \\\hline
DP0 & Data Preview 0 \\\hline
DR2 & Data Release 2 \\\hline
IDF & Interim Data Facility \\\hline
LSST & Legacy Survey of Space and Time (formerly Large Synoptic Survey Telescope) \\\hline
RTN & Rubin Technical Note \\\hline
SQL & Structured Query Language \\\hline
TAP & Table Access Protocol \\\hline
deg & degree; unit of angle \\\hline
\end{longtable}

% If you want glossary uncomment below -- comment out the two lines above
%\printglossaries





\end{document}
